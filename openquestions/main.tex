\documentclass{article}

% Language setting
% Replace `english' with e.g. `spanish' to change the document language
\usepackage[english]{babel}

% Set page size and margins
% Replace `letterpaper' with `a4paper' for UK/EU standard size
\usepackage[letterpaper,top=2cm,bottom=2cm,left=3cm,right=3cm,marginparwidth=1.75cm]{geometry}

% Useful packages
\usepackage{amsmath}
\usepackage{amsfonts}

\usepackage{graphicx}
\usepackage[colorlinks=true, allcolors=blue]{hyperref}
\newtheorem{theorem}{Theorem}
\newtheorem{definition}{Definition}
\title{Open Problems \\Computable Structure Theory and Interactions 2024 \\ TU Wien July
15--17 2024}
\author{Vittorio Cipriani and Dino Rossegger}
\date{}
\begin{document}
\maketitle




\section*{Question 1 {\small{by Dino Rossegger}}}
Suppose $R \subseteq \mathbb{N}^k$ is definable by a $\Sigma_1(X)$ formula in $\mathcal{L}_{\omega_1\omega}$ for some comeager set $X$.

\subsection*{Questions}
\begin{itemize}
    \item Is $R$ $\Sigma_1(\emptyset)$-definable?
    \item What about the same question as above replacing \lq\lq comeager\rq\rq\ by \lq\lq co-null\rq\rq?
\end{itemize}

\subsection*{Answer (Aguilera, Miller, Rossegger, Villano)}
By the relativization of the easy direction of the Ash-Knight-Manasse-Slaman and
Chisholm theorem for $\alpha=1$, if a relation $R$ on a
computable structure $\mathcal A$ is $\Sigma_1^{in}(X)$ definable, then
$R^\mathcal{B}$ is $X$-c.e.\ for every computable copy $\mathcal B$ of $\mathcal
A$. By the folklore (?) result that a set $A\subseteq \omega$ is computably
enumerable if and only if the set $\{ X\subseteq \omega: A\text{ is $X$-c.e.}\}$
is comeagre we obtain that $R^\mathcal{B}$ is c.e.\ for every computable
$\mathcal B\cong \mathcal A$. Applying the Ash-Knight-Manasse-Slaman and
Chisholm theorem we obtain that $R$ is $\Sigma^{in}_1(\emptyset)$-definable.

The same argument works with conull instead of comeagre by replacing the
folklore result in the above argument by an argument of de
Leeuw-Moore-Shannon-Shapiro (see [DH, Theorem 8.12.1]).

[DH] Downey, Rodney G., and Denis R. Hirschfeldt. 2010. Algorithmic Randomness
and Complexity. Springer Science \& Business Media.

\subsection*{Further questions (added post-hoc)}
These results lead to the following questions:
\begin{itemize}
    \item If $R$ is $\Sigma_\alpha^{in}$-definable for a comeagre (conull) set
        of $X$, is $R$ $\Sigma_{\alpha}^{in}(\emptyset)$?
    \item In the above we implicitly allowed parameters in our formulas. If we disallow
parameters we obtain the following version of Ash-Knight-Manasse-Slaman and
Chisholm for $\Sigma_1^{in}$-definability: There is a computable function $f$ such that for
$\Phi_e\cong \mathcal A$, $W_{f(e)}=R^{\Phi_e}$ if and only if $R$ is
$\Sigma_1^{in}$-definable without parameters. 
There are many ways one could try to uniformize the above question. For example,
what if on a comeagre set of reals we have $\Sigma^{in}_1$-definability without
parameters but always with different formulas?
\end{itemize}



\section*{Question 2 {\small{by Russell Miller}}}
\textit{Notations}:
\begin{itemize}
    \item $\mathsf{TFAB}_n$ denotes the class of torsion free abelian group of rank $n$;
    \item $\mathsf{FTD}_n$ denotes the class of fields of trascendence degree $n$;
    \item $\mathsf{TD}_n$  denotes the class of fields of characteristic $0$ and transcendence degree $n$.
\end{itemize}
\textit{Known results}:
\begin{itemize}
    \item (Hjorth, Thomas)  $\mathsf{TFAB}_n<_B \mathsf{TFAB}_{n+1}$ (here $<_B$ denotes Borel reducibility). In contrast, 
    \item (Thomas, Velikovic)   For every $n$, $\mathsf{FTD}_{13} \equiv_B \mathsf{FTD}_n$.
\end{itemize}
The following is known in the context of Turing computable embeddings (denoted by $\leq_{tc}$).
\begin{itemize}
    \item (Ho, Knight, Miller)    For every $n$, $\mathsf{TFAB}_n \leq_{tc} \mathsf{FTD}_n$.
\end{itemize}
\textit{Questions}:
\begin{itemize}
    \item $\mathsf{TD}_n \geq_B \mathsf{FTD}_{n+1}$?
    \item  $\mathsf{TD}_n \leq_B \mathsf{TFAB}_{n}$? The answer is no for $n \geq 13$.
    \item Same question as above but replacing $\leq_B$ with $\leq_{tc}$. Again, the answer is no for $n\geq13$.
\end{itemize}


\section*{Question 3 {\small{by Stefan Vatev}}}
It is known that for all limit ordinals $\alpha,\beta$, $(\alpha,\alpha^*)\equiv_{tc} (\beta,\beta^*)$.\\
\textit{Question}:
\begin{itemize}
    \item What happens if we replace Turing computable embedding with enumeration reducibility?
\end{itemize}

\begin{definition}[Enumeration reducibility]
    Let $A,B \subseteq \mathbb{N}$. We say that $A$ is enumeration reducible to $B$ (notation, $A \leq_e B$) if:
    \[(\exists e)(x \in A \iff (\exists \text{ finite } D\subseteq B)(\langle x,D\rangle \in W_e)).\]
\end{definition}
For example, $(\omega\cdot n, \omega^* \cdot n) \leq_c (\omega \cdot k, \omega^* \cdot k) $ if and only if $n | k$ (here $\leq_c$ denotes computable embedding).\\


\section*{Question 4 {\small{by Gianluca Paolini}}}

\begin{enumerate}
    \item Is there a computable first-order theory $T$ such that the isomorphism relation for models of $T$ with domain $\omega$ is as complicated as graph isomorphism with respect to Borel reducibility?
    \item Consider classes where we know that embeddability is not an analytic
        complete quasi-order such as linear orderings or Boolean algebras. Is
        elementary embeddability for this classes analytic complete?
\end{enumerate}

\subsection*{Answer to (1)}
Yes! We can obtain a reduction $f:Graphs \to Graphs$ such that for all graphs
$x,y$, $f(x)\equiv f(y)$. Essentially, we just have to use the
pairs-of-structure theorem of Ash and Knight and do a clever jump inversion. For
example the construction given in [Ros] already has the above property. The
 proof of Lemma 7 in [Ros] already shows mutatis mutandis that $f(x)\equiv f(y)$
 for the reduction provided there.

 [Ros] Rossegger, Dino. 2022. “Degree Spectra of Analytic Complete Equivalence Relations.” The Journal of Symbolic Logic 87 (4): 1663–76. https://doi.org/10.1017/jsl.2021.82.


\section*{Question 5 {\small{by Mateusz \L{}e\l{}yk}}}
Let the Scott function of a theory  be the function defined as 
 \[S_T(\alpha):=|\{\mathcal{M} \models T : SR(\mathcal{M})= \alpha\}|.\] 
 \textit{Question}:
 \begin{itemize}
     \item Is there a non-trivial first-order theory $T$ such that $S_T$ is not monotone or has interesting behavior?
 \end{itemize}



\section*{Question 6 {\small{by David Gonzalez}}}
What are the possible Scott spectra of first-order theories?

\end{document}
